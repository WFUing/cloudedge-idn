请你修改下面的段落,用latex代码形式给出,修改的要求在文中用括号给出,同时你要扮演云边协同和软件工程的专家,同时深谙人类写论文的方式,有严谨的学术性,不要出现AI常犯的表达重复、逻辑僵化的问题



括号中如果不是修改要求,是正文中解释说明的内容,尽量不要在括号中写),写在正文中


算法通过两阶段方法动态寻找最优批处理规模 $B_{opt}$ 和对应的单数据平均处理时间 $t_{avg\_opt}$。第一阶段采用指数增长的方式快速探测系统的最大承载能力,初始批量设为 $B_{init}$,每次加倍直到出现资源过载或性能下降;第二阶段利用二分查找进一步精确定位最优解。输入包括待测试的模型 $Model$、输入样例 $Input$ 和初始批量 $B_{init}$,输出为最优批处理大小 $B_{opt}$ 和最优单数据平均处理时间 $t_{avg\_opt}$。

如何通过动态资源感知和优化调度方法,在云端与边缘端之间实现高效的资源分配,以提升系统的实时性和稳定性?

然而,K3s 缺乏专门的云边协同组件,而 KubeEdge 和 OpenYurt 尽管引入了更精细的节点分组与管理功能,

请你根据上面的模型设计,在edgecloudsim中实现仿真实验 先说明现有的edgecloudsim提供了哪些东西,我具体可以做什么实验 我想实现仿真实验,仿真满足用户可以自己定义多少个边缘节点,多少个云端节点,多少个边缘集群(按照边缘节点数量自动分配边缘节点);设备周期性采集,要实现采集的频率可以变化,从而查看调度策略的稳定性;AI负载单次运行时间要符合云边节点的特性  1. 证明有这个调度模型和使用轮询等方式判断调度目标。  2. 证明有边缘集群和没有边缘集群元数据交互效率的比较  3. 证明边缘集群调度器中队列按单次采集量大小排序和没有排序的区别(你可以极端一点反过来排序给出结果)  4. 证明云端调度器中队列按最少任务先调度,最后设备成功调度的量与没有这样调度的区别你可以极端一点反过来排序给出结果)

\item \textbf{动态 QoS 需求不适应}:实际应用中的服务质量需求往往是动态变化的,可能遇到的网络波动、节点掉线等情况。因此,模型分配策略需要能够根据实时变化的应用场景,动态调整深度学习模型在云端和边缘节点之间的部署位置,以最优地满足多维度的服务质量需求。