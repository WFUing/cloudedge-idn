
\chapter{云边协同下深度学习推理服务的调度方法}

\section{推理服务网络模型}

\subsection{模型定义}

\subsection*{1. 节点定义}

\subsubsection*{1.1 云端节点}
设云端节点集合为 \( C = \{c_1, c_2, \dots, c_m\} \)。每个云端节点 \( c_i \) 定义如下:
\begin{itemize}
    \item \(\text{CPU}(c_i)\):云端节点 \( c_i \) 的 CPU 容量。
    \item \(\text{MEM}(c_i)\):云端节点 \( c_i \) 的内存容量。
    \item \(\text{GPU}(c_i)\):云端节点 \( c_i \) 的 GPU 容量。
    \item \(\text{BW}(c_i)\):云端节点 \( c_i \) 的带宽容量(单位:数据大小/时间)。
\end{itemize}

\subsubsection*{1.2 边端节点}
设边端节点集合为 \( E = \{e_1, e_2, \dots, e_n\} \)。每个边端节点 \( e_j \) 定义如下:
\begin{itemize}
    \item \(\text{CPU}(e_j)\):边端节点 \( e_j \) 的 CPU 容量。
    \item \(\text{MEM}(e_j)\):边端节点 \( e_j \) 的内存容量。
    \item \(\text{BW}(e_j)\):边端节点 \( e_j \) 的带宽容量。
    \item \(\text{GPU}(e_j) = 0\):边端节点没有 GPU 资源。
\end{itemize}

\subsection*{2. 用户请求定义}
设用户请求集合为 \( R = \{r_1, r_2, \dots, r_k\} \)。每个用户请求 \( r_i \) 定义如下:
\begin{itemize}
    \item \(\text{ID}(r_i)\):请求的唯一标识符。
    \item \(\text{Type}(r_i)\):请求的类型,表示需要处理的 AI 任务类型。
    \item \(\text{Duration}(r_i)\):请求的运行周期时长(单位:时间)。
    \item \(\text{DataCount}(r_i)\):请求包含的数据数量。
    \item \(\text{DataSize}(r_i)\):请求的数据总大小(单位:数据大小)。
    \item \(\text{Accuracy}(r_i)\):请求要求的最低准确率。
    \item \(\text{Source}(r_i)\):请求的节点来源(可以是某个边端节点 \( e_j \))。
\end{itemize}

\subsection*{3. AI 负载定义}
设 AI 负载集合为 \( A = \{a_1, a_2, \dots, a_p\} \)。每个 AI 负载 \( a_i \) 定义如下:
\begin{itemize}
    \item \(\text{ID}(a_i)\):AI 负载的唯一标识符。
    \item \(\text{Name}(a_i)\):AI 负载的名称。
    \item \(\text{SupportedTypes}(a_i)\):能够响应的请求类型集合。
    \item \(\text{Latency\_GPU}(a_i, y)\):在 GPU 环境下,节点 \( y \) 上运行该负载的单次推理时延(单位:时间)。
    \item \(\text{Latency\_NoGPU}(a_i, y)\):在非 GPU 环境下,节点 \( y \) 上运行该负载的单次推理时延(单位:时间)。
    \item \(\text{Throughput}(a_i)\):单次推理的吞吐量(每次推理能处理的图片数量)。
    \item \(\text{Resources}(a_i) = (\text{CPU}(a_i), \text{MEM}(a_i), \text{GPU}(a_i))\):运行所需的资源量。
\end{itemize}

\subsection*{4. 调度目标与约束}

\subsubsection*{4.1 调度决策变量}
定义调度变量:
\begin{itemize}
    \item \( x_{ij} \geq 0 \):表示请求 \( r_i \) 中由 AI 负载 \( a_j \) 处理的数据比例(\(0 \leq x_{ij} \leq 1\),所有负载的比例之和为 1)。
    \item \( y_{ij} \in C \cup E \):表示请求 \( r_i \) 中分配给 AI 负载 \( a_j \) 的任务被分配到的节点(云端节点或边端节点)。
\end{itemize}

约束:
\begin{itemize}
    \item 数据分配总量约束:请求 \( r_i \) 中,分配给所有 AI 负载的数据比例之和等于 1,即:
    \[
    \sum_{j} x_{ij} = 1, \quad \forall r_i \in R
    \]
    其中,\( x_{ij} \) 表示负载 \( a_j \) 处理的比例。
\end{itemize}

\subsubsection*{4.2 约束条件}

\paragraph*{1. 运行时长约束}


\paragraph*{1. 运行时长约束}

请求 \( r_i \) 在运行周期内必须完成其所有数据的处理,约束如下:

\[
\sum_{j} \left( \frac{x_{ij} \cdot \text{DataCount}(r_i)}{\text{Throughput}(a_j)} \cdot \text{Latency}(a_j, y_{ij}) \right) + \frac{\text{DataSize}(r_i)}{\text{BW}(y_{ij})} \leq \text{Duration}(r_i), \quad \forall r_i \in R
\]

其中:
\begin{itemize}
    \item \( x_{ij} \cdot \text{DataCount}(r_i) \):表示负载 \( a_j \) 处理的具体数据量。
    \item \( \text{Throughput}(a_j) \):AI 负载 \( a_j \) 的单次推理吞吐量。
    \item \( \text{Latency}(a_j, y_{ij}) \):AI 负载 \( a_j \) 在节点 \( y_{ij} \) 上的单次推理时延。
    \item \( \frac{\text{DataSize}(r_i)}{\text{BW}(y_{ij})} \):数据传输时延,取决于请求的数据大小和节点 \( y_{ij} \) 的带宽。
    \item \( \text{Duration}(r_i) \):请求的运行周期时长。
    \item \[
            \text{Latency}(a_j, y_i) =
            \begin{cases} 
            \text{Latency\_GPU}(a_j), & \text{if } \text{GPU}(y_i) > 0 \\
            \text{Latency\_NoGPU}(a_j), & \text{if } \text{GPU}(y_i) = 0
            \end{cases}
        \]
\end{itemize}

该约束确保所有分配给负载的任务在请求时长内能够完成。

\paragraph*{2. 准确率约束}

请求的处理结果需满足其准确率要求,考虑数据分配比例后,约束如下:
\[
\sum_{j} x_{ij} \cdot \text{Accuracy}(a_j) \geq \text{Accuracy}(r_i), \quad \forall r_i \in R
\]
其中:
\begin{itemize}
    \item \( x_{ij} \cdot \text{Accuracy}(a_j) \):表示负载 \( a_j \) 处理的比例对应的贡献准确率。
    \item \(\text{Accuracy}(r_i)\):请求 \( r_i \) 对整体任务的最低准确率要求。
\end{itemize}

\paragraph*{3. 资源容量约束}
每个节点的资源总使用量不能超过其可用容量:
\[
\sum_{r_i \in R} x_{ij} \times \text{Resources}(a_j) \leq \text{AvailableResources}(y_i), \quad \forall y_i \in C \cup E
\]

\subsubsection*{4.3 优化目标}

优化目标分为两级优先级:

\paragraph*{1. 一级优化目标:最大化成功响应的请求数量}

目标是优先最大化成功响应的请求数量。定义一个二进制变量:
\[
z_i = 
\begin{cases} 
1, & \text{如果请求 } r_i \text{ 被成功响应(满足所有约束)} \\
0, & \text{否则}
\end{cases}
\]
一级优化目标为:
\[
\max \sum_{r_i \in R} z_i
\]

\paragraph*{2. 二级优化目标:在满足一级目标的基础上最小化带宽利用}

在一级目标下,进一步最小化所有请求的数据传输所消耗的带宽:
\[
\min \sum_{r_i \in R} \sum_{j} \frac{x_{ij} \cdot \text{DataSize}(r_i)}{\text{BW}(y_{ij})}
\]
其中:
\begin{itemize}
    \item \( x_{ij} \cdot \text{DataSize}(r_i) \):表示负载 \( a_j \) 处理的任务对应的数据大小。
    \item \( \text{BW}(y_{ij}) \):节点 \( y_{ij} \) 的带宽。
\end{itemize}





\subsection{推理服务调度问题的形式化表述}

\section{云边协同下的推理服务调度方法}




主动探测与被动采集结合的方法